\documentclass[10pt,a4paper]{article}

\usepackage{ctex}  % 支持中文
\usepackage{graphicx} % 插入图片宏包
\usepackage{listingsutf8} %插入代码段
\usepackage{color} %颜色
\definecolor{dkgreen}{rgb}{0,0.6,0}
\definecolor{gray}{rgb}{0.5,0.5,0.5}
\definecolor{mauve}{rgb}{0.58,0,0.82}
\usepackage{latexsym}
\usepackage{amsmath}
\usepackage{CJK}
\usepackage{CJKnumb}
\usepackage{tikz}
\usepackage{txfonts}
\usepackage{bm}
\usepackage{indentfirst}
\usepackage{pifont}
\usepackage{cases}
\usepackage{fontspec} %字体
\usepackage{courier} %字体
\usepackage{float}
\usepackage{gensymb}
\usepackage[CJKbookmarks=true]{hyperref} % 目录与章节链接
\hypersetup{hidelinks} % 取消链接方框
%\setmonofont{Consolas} %设置全局字体
\setmonofont{Courier}  %mac os  字体
\usepackage{pdflscape}
\usepackage{multicol}

%%%%%%%%%%% CJK下设置中文字体 %%%%%%%%%%%%%
\newcommand{\song}{\CJKfamily{song}}    % 宋体   (Windows自带simsun.ttf)
\newcommand{\fs}{\CJKfamily{fs}}        % 仿宋体 (Windows自带simfs.ttf)
\newcommand{\kai}{\CJKfamily{kai}}      % 楷体   (Windows自带simkai.ttf)
\newcommand{\hei}{\CJKfamily{hei}}      % 黑体   (Windows自带simhei.ttf)
\newcommand{\li}{\CJKfamily{li}}        % 隶书   (Windows自带simli.ttf)
\usepackage{geometry} % 设置页边距的包
\usepackage{titlesec} %设置页眉页脚的包
\geometry{left = 2cm, right = 2cm, top = 2.5cm, bottom = 2.5cm}  % 设置页边距
\usepackage{lastpage}%获得总页数
\usepackage{fancyhdr}
\pagestyle{fancy}

%%%%%%%%%%%  设置字体大小 %%%%%%%%%%%%%
\newcommand{\chuhao}{\fontsize{42pt}{\baselineskip}\selectfont}
\newcommand{\xiaochuhao}{\fontsize{36pt}{\baselineskip}\selectfont}
\newcommand{\yihao}{\fontsize{28pt}{\baselineskip}\selectfont}
\newcommand{\erhao}{\fontsize{21pt}{\baselineskip}\selectfont}
\newcommand{\xiaoerhao}{\fontsize{18pt}{\baselineskip}\selectfont}
\newcommand{\sanhao}{\fontsize{15.75pt}{\baselineskip}\selectfont}
\newcommand{\sihao}{\fontsize{14pt}{\baselineskip}\selectfont}
\newcommand{\xiaosihao}{\fontsize{12pt}{\baselineskip}\selectfont}
\newcommand{\wuhao}{\fontsize{10.5pt}{\baselineskip}\selectfont}
\newcommand{\xiaowuhao}{\fontsize{9pt}{\baselineskip}\selectfont}
\newcommand{\liuhao}{\fontsize{7.875pt}{\baselineskip}\selectfont}
\newcommand{\qihao}{\fontsize{5.25pt}{\baselineskip}\selectfont}

\setlength{\parindent}{2em}                 % 首行两个汉字的缩进量
\setlength{\parskip}{1pt plus1pt minus1pt} % 段落之间的竖直距离
\renewcommand{\baselinestretch}{1.0}        % 定义行距
\setlength{\abovedisplayskip}{1pt plus1pt minus1pt}     %公式前的距离
\setlength{\belowdisplayskip}{1pt plus1pt minus1pt}     %公式后面的距离
\setlength{\arraycolsep}{2pt}   %在一个array中列之间的空白长度, 因为原来的太宽了


%\setlength{\parindent}{2em}                 % 首行两个汉字的缩进量
%\setlength{\parskip}{3pt plus1pt minus1pt} % 段落之间的竖直距离
%\renewcommand{\baselinestretch}{1.2}        % 定义行距
%\setlength{\abovedisplayskip}{2pt plus1pt minus1pt}     %公式前的距离
%\setlength{\belowdisplayskip}{6pt plus1pt minus1pt}     %公式后面的距离
%\setlength{\arraycolsep}{2pt}   %在一个array中列之间的空白长度, 因为原来的太宽了

\lstset{
		columns=fixed,       
		numbers=left,                                        % 在左侧显示行号
		frame=leftline,                                          % 行号与代码之间显示一根竖线
		breaklines = true,                 % 代码过长自动换行
		tabsize = 4,
		backgroundcolor=\color{white},            % 设定背景颜色
		numberstyle=\xiaosihao\color{black},                       % 设定行号格式
%		keywordstyle=\wuhao\color{black},                 % 设定关键字颜色
		basicstyle = \ttfamily\wuhao\color{black},       % 设置基本字体格式
%		commentstyle=\ttfamily\wuhao\it\color{dkgreen},                % 设置代码注释的格式\label{key}
%		stringstyle=\rmfamily\slshape\color{red},   % 设置字符串格式
		showstringspaces=false,                              % 不显示字符串中的空格
		language=c++,                                        % 设置语言 
		morekeywords={alignas,continute,friend,register,true,alignof,decltype,goto,  % 增加keywords
		    reinterpret_cast,try,asm,defult,if,return,typedef,auto,delete,inline,short,
		    typeid,bool,do,int,signed,typename,break,double,long,sizeof,union,case,
		    dynamic_cast,mutable,static,unsigned,catch,else,namespace,static_assert,using,
		    char,enum,new,static_cast,virtual,char16_t,char32_t,explict,noexcept,struct,
		    void,export,nullptr,switch,volatile,class,extern,operator,template,wchar_t,
		    const,false,private,this,while,constexpr,float,protected,thread_local,
		    const_cast,for,public,throw,std, ll},
		    emph={map,set,multimap,multiset,unordered_map,unordered_set,
		    unordered_multiset,unordered_multimap,vector,string,list,deque,
		    array,stack,forwared_list,iostream,memory,shared_ptr,unique_ptr,
		    random,bitset,ostream,istream,cout,cin,endl,move,default_random_engine,
		    uniform_int_distribution,iterator,algorithm,functional,bing,numeric,vector,},
}


%以下命令中L--左侧 R--右侧 C--中间 O--奇数页 E--偶数页
\fancyfoot[CO,RE]{}%奇数页中间,偶数页右侧页脚为空
\fancyfoot[LO,RE]{TI1050}%奇数页左侧,偶数页中间页脚为空
\fancyfoot[RO,LE]{\thepage\ of
\pageref{LastPage}}%奇数页右侧,偶数页左侧显示 当前页 of 总页数
\renewcommand{\headrulewidth}{0.5pt}%改为0pt即可去掉页眉下面的横线
\renewcommand{\footrulewidth}{0.5pt}%改为0pt即可去掉页脚上面的横线


\newcommand*{\circled}[1]{\lower.7ex\hbox{\tikz\draw (0pt, 0pt)%
    circle (.5em) node {\makebox[1em][c]{\small #1}};}}